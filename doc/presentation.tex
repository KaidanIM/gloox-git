

% presentation
% ------------
% 
% slide 1:
% --------
% 
% title, name, thanks
% 
% 
% % greeting, name, purpose of being here
% %
% % introduction
% % ------------
% % - need for lots of processing power to calculate some data
% % - often, data format is specific and fixed
% % --> no need for a dedicated and universally usable cluster
% % - popular example: Seti@Home
% % --> important: volatile clientele, easy deployment
% 
% % Aim: to create a set of software tools to easily set up a dynamic cluster
% %      that takes care of distributing jobs to a number of clients and collecting
% %      the corresponding results.
% 
% 
% 
% slide 2:
% --------
% 
% rough architecture of framework
% 
% 
% % - to achieve the aim, this system was developed
% % explanation of architecture
% % components an their purpose
% 
% 
% 
% slide 3:
% --------

% $Header: /cvsroot/latex-beamer/latex-beamer/solutions/generic-talks/generic-ornate-15min-45min.en.tex,v 1.4 2004/10/07 20:53:08 tantau Exp $

\documentclass{beamer}
\mode<presentation>
\usepackage[T1]{fontenc}
\usepackage[latin1]{inputenc}
\usepackage[english]{babel}
\usepackage{times}

\title{A Software Framework for Dynamic Clusters}
\subtitle{Final Year Project 2004/2005}

\author{Jakob Schr�ter}

\date{16/03/2005}

\subject{Framework for Dynamic Clusters}


\begin{document}

\begin{frame}
  \titlepage
\end{frame}

\begin{frame}
  \frametitle{Outline}
  \tableofcontents
\end{frame}



\section{Introduction}

\subsection{Extensible Message and Presence Protocol}
\begin{frame}[fragile]
  \frametitle{Extensible Message and Presence Protocol}
  \begin{itemize}
    \item Based on the Extensibel Markup Language (XML)
    \item Streaming Protocol
    \item Relative to HTML
  \end{itemize}
\ \ \ \ Example:
  \begin{verbatim}
    <presence from="jid@server.com"
              to="other@server.com">
      <status>
        online
      </status>
    </presence>
  \end{verbatim}
\end{frame}


\section{Architecture}
\begin{frame}
  \frametitle{Architecture}
  \begin{center}
    \includegraphics{fyp.0}
  \end{center}
\end{frame}


\section{Components}
\begin{frame}
  \frametitle{Components}
  \begin{center}
    \includegraphics{fyp.8}
  \end{center}
\end{frame}

\subsection{XMPP Client Library}
\begin{frame}
  \frametitle{XMPP Client Library}
  \begin{itemize}
    \item Encapsulates XMPP Communication
    \item Uses Observer Pattern
    \item Reusable
  \end{itemize}
\end{frame}

\subsection{Feeder Library}
\begin{frame}
  \frametitle{Feeder Library}
  \begin{itemize}
    \item Keeps Track of Connected Workers
    \item Distributes Data According to Availability
  \end{itemize}
  \addvspace{1cm}
  \begin{center}
    \includegraphics{fyp.9}
  \end{center}
\end{frame}

\subsection{Worker Library}
\begin{frame}
  \frametitle{Worker Library}
  \begin{itemize}
    \item Registers with the Feeder
  \end{itemize}
  \addvspace{1cm}
  \begin{center}
    \includegraphics{fyp.10}
  \end{center}
\end{frame}


\section*{Summary}
\begin{frame}
  \frametitle<presentation>{Summary}
  \begin{itemize}
    \item Finished Implementation of Core Libraries
    \item XMPP Client Library Used in Two More Applications
  \end{itemize}

  % The following outlook is optional.
  \vskip0pt plus.5fill
  \begin{itemize}
  \item
    Remaining Work
    \begin{itemize}
      \item Example Implementations
      \item Statistics Viewer
      \item Documentation
    \end{itemize}
  \end{itemize}
\end{frame}


\end{document}


