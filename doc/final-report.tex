\documentclass[english,a4paper,12pt]{article}
\usepackage[latin1]{inputenc}
\usepackage[T1]{fontenc}
\usepackage[english]{babel}
\usepackage[lsbu]{harvard}
\usepackage{graphicx}
\usepackage{fancyhdr}
\usepackage{ltablex}
\usepackage{booktabs}
\usepackage{helvet}
\usepackage{moreverb}
\usepackage{float}
\usepackage{lastpage}
\usepackage{appendix}
\usepackage[ps2pdf]{hyperref}
%\usepackage{tocloft}

% setup hyperref
\hypersetup{pdftitle={XMPPGrid: A Grid Framework},breaklinks=true,colorlinks=true}

% redefine from "URL:"
\def\harvardurl{[Online] Availabe at }
\def\harvardaccessed{accessed }
% remove leading "References" from the bibliography
\makeatletter
  \def\thebibliography#1{\list
      {[\arabic{enumi}]}{\settowidth\labelwidth{[#1]}\leftmargin\labelwidth
        \advance\leftmargin\labelsep
        \usecounter{enumi}}
        \def\newblock{\hskip .11em plus .33em minus .07em}
        \sloppy\clubpenalty4000\widowpenalty4000
        \sfcode`\.=1000\relax}
\makeatother

% sans-serife for the whole document
\renewcommand*{\familydefault}{\sfdefault}

% set up header and footer
%\pagestyle{fancy}
\fancyhead{}
%\fancyhead[RE]{Framework for Distributing Data using XMPP}
%\fancyhead[LE]{Final Year Project}
%\fancyhead[RO]{Interim Report}
\lhead{Jakob Schr�ter\\
       XMPPGrid: A Grid Framework}
\rhead{Final Year Project 2004/2005\\
       Final Report}
\fancyfoot{}
%\fancyfoot[CO,CE]{\thepage{} of \pageref{LastPage}}
\cfoot{\thepage{} of \pageref{LastPage}}
\setlength{\headheight}{28pt}
%\fancyheadoffset[LE,RO]{\marginparsep+\marginparwidth}

\title{Final Year Project\\
%       Framework for Distributing and Processing Scientifical Data using XMPP \\
       XMPPGrid: A Grid Framework\\
       Final Report}
\author{Jakob Schr�ter\\
        \texttt{js@camaya.net}\\
        Computer Systems and Networks}


\begin{document}
\maketitle
\addvspace{9cm}
This report has been submitted for assessment towards a Bachelor of Engineering Degree in Computer Systems and Networks in the School of Engineering, South Bank University.
This report is written in the authors own words and all sources have been properly cited.

\begin{center}
Author's signature:
\end{center}
\thispagestyle{empty}

\clearpage
\begin{center}
\thanks{\textbf{Acknowledgements}
          \\
          \addvspace{3cm}
          I would like to thank Dr Goran Bezanov for his support throughout the whole project life time.}
\end{center}
\thispagestyle{empty}


\clearpage
\begin{abstract}
Grids provide a means for solving computational expensive tasks in a more reasonable time frame than a single machine can offer. Be it prime factorisation or searching for extraterrestrial intelligence by filtering background noise captured from space, applications for grids are plenty. To make use of a number of computers to process data in a distributed manner, some kind of software is necessary to coordinate the effort.

The Extensible Message and Presence Protocol (XMPP) is currently mostly used in the Instant Messaging domain. But its possible field of operation is much wider. Its extensibility allows for customisation to a great extent.

This project researches the usability of XMPP in a cluster-like environment to distribute data to a grid's nodes. These nodes carry out the actual processing and return the result.

During the life-span of this project jointly acting software components have been described and created to make arbitrary numbers of computers accessible and act cluster-like.
\end{abstract}
\thispagestyle{empty}


\clearpage
\thispagestyle{empty}
\tableofcontents
\thispagestyle{empty}


\clearpage
\pagestyle{fancy}

\section{Introduction}
\paragraph{}
In almost every branch of science data is either collected or created that has to be processed in some way. Sometimes a single machine is unable to keep up with the sheer amount of data and/or the demand for computing power to handle this data in a timely manner. A possible solution to this problem is, of course, to invest in faster hardware. This is a somewhat limited approach. At some point it is simply no longer possible to buy a faster (single-CPU) computer or, more likely for the average company, the costs for acquirement, installation and maintenance are much too high and outweigh the benefits. A cheaper solution would be to utilise a cluster or a grid.
\paragraph{}
A cluster, as \citeasnoun*{wikipedia001} explains, is an arbitrary number of computers, connected over some kind of network and offering some sort of parallelisation. To the user a cluster usually looks like a single computer. The software takes care of distributing tasks, data or algorithms. Actually, according to \citeasnoun*{wikipedia002} and \citeasnoun*{top500}, even today's supercomputers are clusters, with literally thousands of CPUs and specially crafted interconnects to make fast communtication between CPUs and memory possible.
\paragraph{}
Grids differ from clusters in that they are a heterogenous network of computers, while clusters are homogenous. Also, grids are more loosely connected than clusters and can utilise a user's desktop system or office machines \cite{wikipedia005}. To form of a grid, particular software is needed that keeps track of nodes, distributes data and manages the results returned.
\paragraph{}
The Extensible Messaging and Presence Protocol (XMPP) defines a means for interprocess communication based on the Extensible Markup Language (XML). It evolved from the Jabber protocol developed in an Internet community since 1999. In May 2004 Internet Engineering Steering Group (IESG) approved the core parts of XMPP \cite{xmpp-core,xmpp-im} as Proposed Standards and in October 2004 the protocols were published as Request for Comments (RFC).
\paragraph{}
The intention of this project is to utilise XMPP to combine the computing power of a number of networked machines to solve a specific problem. XMPP will be used to control data flow from and to the nodes.


\section{Aim and Objectives}

\subsection{Aims}
\paragraph{}
The main technical aim of this project is to design and implement a prob\-lem-in\-de\-pen\-dent software framework that helps setting up a grid, supporting an arbitrary number of machines. Computationally expensive tasks can then be distributed to the nodes of this grid. Communication of these nodes with the coordinating master will happen by means of the Extensible Message and Presence Protocol (XMPP).
\paragraph{}
Additionally, this project has educational aims. Its purpose is to help the author with getting more comfortable in dealing with larger projects. It serves as an example of work which has to be carried out with only little assistance.

\subsection{Objectives}
\paragraph{}
The project is split into several sub-tasks which mostly depend on one another. The particular dependencies, as fixed for the final schedule, are listed in Figure \ref{tab:fin_actionplan} on page \pageref{tab:fin_actionplan}.

\paragraph{}
The following subsidiary objectives are considered part of the project aim:
\begin{enumerate}
\item \label{item:req} Requirements Specification for the project as a whole
\item \label{item:design} Design of the individual parts of the project (Worker, Feeder, sample implementations, Statistics Viewer, Feeder Front End)
\item \label{item:db} Defining the database structure
\item \label{item:client} Writing the XMPP Client Library
\item \label{item:worker} Writing the Worker Library
\item \label{item:feeder} Writing the Feeder Library
\item \label{item:samplF} Writing a Sample Worker implementation
\item \label{item:samplW} Writing a Sample Feeder implementation
\item \label{item:testing} Testing
\item \label{item:frontend} Writing the Feeder Front End
\item \label{item:viewer} Writing the Statistics Viewer
\item \label{item:manual} Writing User Manuals and generating API documentation
\end{enumerate}
Items \ref{item:req} through \ref{item:testing} are considered core parts of the project. Therefore, most attention will be paid to these tasks. As explained in Section \ref{sec:testing}, the testing will happen in parallel to the programming. There is no extra testing time planned.


\section{Deliverables}
\paragraph{}
The following deliverables result from the afore mentioned aims and objectives.

\paragraph{Interim Report} The Interim Report gives an overview of the project topic, its preliminary status and the project schedule.
\paragraph{XMPP Client Library} The XMPP Client Library implements the necessary functionality to connect to and communicate with an XMPP server.
\paragraph{Feeder Library} The Feeder Library implements the necessary functions to send data packets to clients and receive results.
\paragraph{Worker Library} The Worker Library implements the necessary functions for clients to receive data packets and send back results.
\paragraph{Feeder Sample Implementation} This program serves the Sample Wor\-ker with data to work on.
\paragraph{Worker Sample Implementation} This is a program implementing an algorithm for prime factorisation utilising the Worker Library for receiving working data.
\paragraph{API Documentation of the XMPP Client Library} The documentation for the XMPP Client Library API.
\paragraph{API Documentation of the Worker Library} The documentation to implement a Worker using the Worker Library.
\paragraph{API Documentation of the Feeder Library} The documentation to implement a Feeder using the Feeder Library.
\paragraph{Feeder Front End} The Feeder Front End enables the operator to view real-time statistics of the grid operation.
\paragraph{Web-based Statistics Analysis Tool} This is a web-based tool to view client and feeder statistics such as processed data packets.
\paragraph{User Manual for the Statistics Analysis Tool} The manual for the Sta\-tis\-tics Analysis Tool.
\paragraph{Final Report} The Final Report reports on the final status of the project.


\section{Technical Background}% and Context}
\paragraph{}
The project is based on well-known and well-understood technologies. This section briefly introduces these core technologies.

\subsection{Distributed Computing}
\paragraph{}
Basically, the term Distributed Computing describes a technology where more or less loosely connected machines work together on solving a problem. They do so by each dealing with a small fraction of the complete problem. This, of course, implies that it is possible to split the problem in smaller sub-problems that can be solved individually. There are lots of challenges out there that meet these requirements. Accordingly, there are many projects that make use of clusters to work on such a problem. Usually, a single machine takes care of assigning tasks to the nodes. The processing of the task could then even happen in a disconnected mode. But upon completion of the computing the result is delivered back to the co-ordinating master.
\paragraph{}
Prominent examples are the afore mentioned Seti@Home ...

\paragraph{}
Software solutions used in this field are very problem specific in that they do not allow for solving a completely different one. 

\paragraph{}
Example institutions where clusters could easily be deployed include schools, universities, corporate environments or laboratories. In those locations usually a large number of computers is available which are used only for in the daytime and not at all at night. During those hours of less or no usage these machines can be used as nodes of a cluster. Such a cluster would be static since the machines don't change their membership in the cluster. Another way to form a cluster is to deploy the Internet. Many more people and machines are reachable and can participate.

\subsection{Extensible Markup Language (XML)}
\paragraph{}
The Extensible Markup Language (XML) is an universal means to describe hierarchical data. It is human readable as well as easily machine parsable. It has been developed by the World Wide Web Consortium (W3C) and is a W3C standard. It is a subset of the Standard Generalized Markup Language (SGML).

\paragraph{}
XML alone is not a programming language, neither is it a markup language per se. It rather is a ruleset that defines a way how to describe hierarchical data. A user of XML has to define specific rules for their specific kind of data on their own in a formal way. Such a formal description can, for example, be a Document Type Definition (DTD). It is not absolutely necessary to write down a DTD to use an XML-derived markup language. But using such a formal description, software can validate a given document for its conformity.

\paragraph{}
XML consists of elements, their attributes and character data. An element has an opening tag and a corresponding closing tag. A tag is a charcter string enclosed by \glqq{}<\grqq{} and \glqq{}>\grqq{}. A closing tag additionally has a slash \glqq{}/\grqq{} right after the first \glqq{}<\grqq{}. Attributes are placed as \texttt{Key="Value"} pairs inside a tag. Data is put between opening and closing tags.

\begin{figure}[H]
\begin{listing}{1}
<outer key="value">
  <inner xmlns="namespace" foo="bar">
    character data
  </inner>
  <tag attr="value" />
</outer>
\end{listing}
\caption{XML Example}
\label{fig:XMLexample}
\end{figure}

\paragraph{}
Tags can be nested infinitely and build up hierarchies as long as hierarchy levels are not mixed. In other words, it is not allowed to close an element after another one was openend but not closed. Figure \ref{fig:XMLexample} shows an example excerpt of a hypothetical (valid) XML document.

\paragraph{}
A relatively new feature in XML are namespaces. Since XML instances may contain elements from different vocabularies, namespaces are used to remove ambiguities and distinguish semantically between equally named elements. The special attribute \texttt{xmlns} denotes the namespace. For example, in Figure \ref{fig:XMLexample} the \texttt{<inner>} element belongs to the namespace \texttt{namespace}.

\subsubsection{XML Parser}
\paragraph{}
* DOM
* SAX


\subsection{Extensible Message and Presence Protocol (XMPP)}
\label{sec:xmpp}
\paragraph{}
The Extensible Message and Presence Protocol is developed in an Internet community since 1999\footnote{More on the history of Jabber and XMPP can be found on http://www.xmpp.org/history.html}. It is also known as Jabber, which also is the original name, and its primary purpose was Instant Messaging. With the formation of a dedicated XMPP Working Group within the Internet Engineering Task Force (IETF) the protocol became an Internet Standard in 2004. It is published by IETF as Request for Comments (RFC) memos 3920 and 3921.
\paragraph{}
XMPP is based on XML in that the actual data exchanged between client and server is an XML stream. After a TCP/IP connection is established between client and server (or two servers), a stream opening tag is sent by both parties. Communication then happens by sending so-called stanzas to and fro. Closing the initial stream tag also closes the underlying TCP/IP connection.
\paragraph{}
A stanza in XMPP is exactly one element, consisting of its opening tag, any number of corresponding attributes, nested tags and character data, respectively, and the closing tag. Figure \ref{fig:streamExample} shows an example stream and a stanza.

\begin{figure}[H]
\begin{listing}{1}
<?xml version='1.0'?>
<stream:stream 
      xmlns:stream='http://etherx.jabber.org/streams' 
      xmlns='jabber:client' to='jabber.cc' version='1.0'>
  <presence from="sender@server.com" to="receiver@server.com">
    <status>
      online
    </status>
  </presence>
</stream>
\end{listing}
\caption{XMPP Stream Example}
\label{fig:streamExample}
\end{figure}

\paragraph{}
Line 1 denotes that the following content complies with XML version 1.0, and is mandatory for any XML document. A stream can be seen as a special type of document. The stream is opened in line 2 and the opening tag of a \texttt{presence} stanza is sent in line 3. The stream is closed again in line 10.
\paragraph{}
In the Instant Messaging world, a widely-used feature of XMPP is the roster. Users can authorise other users to allow them to see their online presence. These authorised users are then (usually) added to the authorising users personal contact list. Upon connection of a user, their client sends the current presence status (usually 'online') to the server which then distributes it to authorised contacts in the user's roster.


\section{Development Environment}

\subsection{Operating System}
\paragraph{}
The Operating System the development of this framework takes place on is a home-grown Linux system. During the development cycle it was made sure that the software compiles on other Linux boxes as well. Since the definition of grid includes the heterogenous component, it is important for the software to run on multiple Operating Systems. However, due to lack of access to different platforms, compilation was tested neither on other UNIX boxes nor on the Microsoft Windows platform.

\subsection{Dependencies}
\label{sec:depend}

\subsubsection{Compile-Time Dependencies}
\label{sec:compdepend}
\paragraph{}
The most important dependency is the Iksemel library, written by Gurer Ozen. Iksemel contains XML parsers (DOM as well as SAX) and offers many XMPP related low-level functions. It was chosen as parsing library for this project due to its efficiency and simplicity of usage, the small size and its self-containment.

\paragraph{}
Early versions of the XMPP Client Library developed in this project depended on the Iksemelmm library which is a C++ wrapper around Iksemel. To reduce the number of external dependencies, the most important parts (namely the class \texttt{Tree}, \texttt{Parser} and \texttt{Stream}) of it where added to the XMPP Client Library. They are discussed in Section \ref{sec:clientlib}.


\subsubsection{Run-Time Dependencies}
\paragraph{XMPP Server}
The central workhorse, as far as the communication between Worker and Feeder is concerned, is the XMPP server. For this project the Jabberd2\footnote{Jabberd2 has its home at \href{http://jabberd.jabberstudio.org}{http://jabberd.jabberstudio.org}} server was chosen. Jabberd2 is an open source XMPP server implementation that (slowly) aims at being completely compliant to the XMPP specification.
\paragraph{}
The XMPP server is responsible for authenticating users and routing stanzas between clients in an XMPP/Jabber network. As mentioned in Section \ref{sec:xmpp}, it also distributes presence information to authorised contacts.

\paragraph{Database}
As a database backend, storing statistical information about the tasks handed out to the Workers, MySQL has been chosen because it is already in use on the machine acting as master in the development infrastructure. Since the storage mechanism is implemented in a modular way, it would be easy to add a PostgreSQL or file-based storage backend.

\paragraph{Webserver}
The webserver running the Statistics Viewer is, in the development infrastructure, Apache httpd. Again, it is used because it already runs on the master machine. In a later production environment, Apache httpd is not necessarily required since PHP, which is used for the implementation of the Statistics Viewer, runs on a large range of webservers.


\subsection{Source Code Versioning}
\paragraph{}
For software developers it is very important to have full control over source code they have written. This becomes more important with a higher number of developers involved in a single project. In such a case usually a Revision Control System (RCS) is used to keep track of code changes. But even for a one-person project a RCS has several advantages, including, but not limited to:

\begin{itemize}
\item Backup, if revision control system is kept on a different machine
\item History of modifications including log messages
\item Recovery of older versions
\item Branching and merging of changes
\end{itemize}

\paragraph{}
From the very first this project used a RCS called Subversion\footnote{Subversion lives at \href{http://subversion.tigris.org}{http://subversion.tigris.org}}. It has been choosen due to positive experience in earlier and current projects and because the necessary infrastructure was already in place.

\paragraph{}
The Subversion repository for this project can be found at \\ \texttt{\href{http://svn.camaya.net/uni/fyp-lsbu/}{http://svn.camaya.net/uni/fyp-lsbu/}}. Only the most current version is accessible by means of HTTP. Older revisions can be accessed using an appropriate Subversion client. The repository hosts the complete source code and all the documentation written during the project's life-span.


\subsection{Build System}
\paragraph{}
The Build System helps users and developers building binary executables as well as documentation from source code. The responsibilities of the Build System include:

\begin{itemize}
\item Checking for availability of dependencies
\item Compiling source files using an appropriate compiler
\item Installing executables and data files to their destination directories
\end{itemize}

\subsubsection{Source Code Compilation}
\paragraph{}
For this project a combination of the tools autoconf\footnote{Autoconf's homepage is at \href{http://www.gnu.org/software/autoconf/}{http://www.gnu.org/software/autoconf/}} and automake\footnote{Automake's homepage is at \href{http://www.gnu.org/software/automake/}{http://www.gnu.org/software/automake/}} is used. Of course, the GNU Compiler Collection  takes care of the final compilation.

\paragraph{autoconf}
Autoconf is used to create a \texttt{configure} script. This contains various checks for more or less obvious dependencies, including those listed in Section \ref{sec:depend}.

\paragraph{automake}
Automake is used to create target platform-specific \texttt{Makefile} templates (called \texttt{Makefile.in}) from their corresponding generic templates (called \texttt{Makefile.am}). The \texttt{configure} shell script creates the final \texttt{Makefile}s from these \texttt{Makefile.in}s. A \texttt{Makefile} contains a set of rules to build the source code, i.e. the actual compile commands.

\subsubsection{API Documentation}
\paragraph{}
The API documentation is created by Doxygen\footnote{Doxygen's homepage is at \href{http://www.doxygen.org}{http://www.doxygen.org}}. It extracts comments from the source files and builds a nice documentation in HTML format from it.


\subsubsection{Project Documentation}
\paragraph{}
The project documentation is built using \LaTeX, BibTeX and MetaPost.


\section{Technical Approach}

\subsection{Architecture}

\begin{figure}[H]
\begin{center}
\includegraphics{fyp.0}
\end{center}
\caption{Project Architecture}
\label{fig:architecture}
\end{figure}

\paragraph{}
Figure \ref{fig:architecture} shows the project's architecture. On one hand side, a Feeder is responsible for creating or gathering data that has to be processed. This data is sent to one of the waiting Workers at a time via the XMPP server. The Workers, on the other hand side, do the necessary processing and return the result to the Feeder. This result is stored in a database. The Statistics Viewer can be used to view statistics about the returned results and the grid operation.

\paragraph{}
Feeder and Worker are each split in two main components. This is shown in Figures \ref{fig:feeder} and \ref{fig:worker}, respectively.

\begin{figure}[H]
\begin{center}
\includegraphics{fyp.9}
\end{center}
\caption{The Feeder in Detail}
\label{fig:feeder}
\end{figure}

\paragraph{}
The Feeder Library communicates with the XMPP Server and the Worker on behalf of the Data Gatherer. Likewise does the Worker Library take care of the XMPP communication for the Algorithm Implementation. Data Gatherer as well as Algorithm Implementation know nothing about the wherefrom and whereabout of the data, respectively.

\begin{figure}[H]
\begin{center}
\includegraphics{fyp.10}
\end{center}
\caption{The Worker in Detail}
\label{fig:worker}
\end{figure}

\paragraph{}
From the previous figures it is clear that both Worker and Feeder are likely to contain similar code which could be shared. Since the connectivity functionality is identical for both of the higher level components and is self-contained, it was decided to write a generic XMPP client library that can even be (and by the time of this writing is) used in other projects. Figure \ref{fig:justification} shows the collaboration of Feeder Library, Worker Library and XMPP Client Library.

\begin{figure}[H]
\begin{center}
\includegraphics{fyp.8}
\end{center}
\caption{Usage of the XMPP Client Library}
\label{fig:justification}
\end{figure}

\paragraph{}
Both higher level libraries base their entire network-based communication on the XMPP Client Library. Receiving and sending data is completely transparent. The XMPP Client Library could be replaced by another library implementing any other means of communication without hassle.

\subsection{Addressing of Workers}
\paragraph{}
The Feeder needs to be able to distinguish between different Workers in order to keep track of them and sending work data to a specific one. It does so by keeping a map of addresses of connected Workers and their respective current status.

\subsubsection{Node Addressing in XMPP}
\paragraph{}
In general, an address in XMPP (also known as Jabber ID or JID) is built using the following syntax:
\begin{center}
\texttt{[\{local\}@]\{server\}[/\{resource\}]}
\end{center}
An addressable node can be either of the following:
\begin{description}
\item[client] with an address of the form: \texttt{\{local\}@\{server\}/\{resource\}}
\item[component] with an address of the form: \texttt{\{local\}@\{server\}} 
\item[server] with  an address of the form: \texttt{\{server\}}
\end{description}
\paragraph{}
For addressing, the resource part is optional under certain circumstances and an entity is always reachable by addressing the bare (without resource part) Jabber ID. Even though an XMPP address looks a lot like an ordinary email address, the XMPP specification allows for more characters to be used in the local part, even whitespace is valid. The server part depends on domain naming rules valid in the Internet. This addressing scheme, including the range of valid characters in an XMPP address, is decribed in greater detail in RFC 3920: XMPP Core \cite{xmpp-core}.

\subsubsection{Addressing in XMPPGrid}
\label{sec:addressing}
\paragraph{}
The design of Worker and Feeder Libraries does not impose any limit whatsoever on the addresses of Workers besides those already defined in RFC 3920. This means it is even possible to have a Worker use the same Jabber ID like a users Instant Messaging account, except for the resource part. An example: User Joe has got an account on server \texttt{example.org}. Their client adds a resource of \texttt{JoesChatClient}. The resulting JID, under which Joe is reachable for chatting, would be:
\begin{center}
\texttt{joe@example.org/JoesChatClient}
\end{center}
If Joe chose to have a Worker use the same address, the full JID would look like:
\begin{center}
\texttt{joe@example.org/JoesWorker}
\end{center}
Following this scheme, it is possible to run a whole grid using only a single account. Each Worker would simply add a different resource part to the bare address to form its unique JID.
\paragraph{}
Further advantages of using XMPP are discussed in Section \ref{sec:extensibility}.

\subsection{The XMPP Client Library}
\label{sec:clientlib}
\paragraph{}
The XMPP Client Library consists of one main class and several helper classes. The main class, \texttt{JClient}, is derived from the stream handling class \texttt{Stream}, which in turn is derived from the XML parser interface \texttt{Parser} and an XML tree abstraction, \texttt{Tree}. The inheritance diagram for \texttt{JClient} is shown in Figure \ref{fig:inhjclient}.

\begin{figure}[H]
\begin{center}
\includegraphics{fyp.11}
\end{center}
\caption{Inheritance Diagram for Class \texttt{JClient}}
\label{fig:inhjclient}
\end{figure}

\paragraph{}
The classes \texttt{Tree}, \texttt{Parser} and \texttt{Stream} were taken from the Iksemelmm\footnote{Originally written by Igor Goryachieff \texttt{<igor@jahber.org>}} library and modified slightly. They live in the \texttt{Iksemel} namespace and form an object-oriented wrapper interface to the Iksemel library. The build system creates a so-called convenience library from them which is linked into the final XMPP Client Library.

\paragraph{}
The most simple example of using the XMPP Client Library would look like shown in Figure \ref{fig:usageExample}.

\begin{figure}[H]
\begin{listing}{1}
void Class::doIt()
{
  JClient* j = new JClient( "user", "password",
                            "server", "resource" );
  j->connect();
}
\end{listing}
\caption{Simple XMPP Client Library Usage Example}
\label{fig:usageExample}
\end{figure}

\paragraph{}
In this fictitious function \texttt{Class::doIt()} a new \texttt{JClient} object is created and initialised with the account's credentials necessary for connecting and authenticating to an XMPP server (lines 3 and 4). As a consequence, the constructor of \texttt{JClient} as shown in Figure \ref{fig:JClientConstructor} is called.

\begin{figure}[H]
\begin{listing}{1}
JClient::JClient( const std::string& username,
                  const std::string& password,
                  const std::string& server,
                  const std::string& resource, int port )
  : m_username( username ), m_password( password ),
  m_server( server ), m_resource( resource ),
  m_port( port ), m_thread( 0 ),
  m_tls( true ), m_sasl( true ),
  m_autoPresence( false ), m_manageRoster( true ),
  m_handleDisco( true ), m_idCount( 0 ), m_roster( 0 ),
  m_disco( 0 )
{
  init();
}
\end{listing}
\caption{Constructor of class JClient (simplified)}
\label{fig:JClientConstructor}
\end{figure}

\paragraph{}
Basically, a number of member variables are initialised (lines 5 to 11). Most importantly, Transport Layer Security\footnote{Encryption of the communication channel.} (TLS) \cite{rfc2246} and Simple Authentication and Security Layer\footnote{Secure Authentication even without encrypted communication channel.} (SASL) \cite{rfc2222} are enabled (line 8). Additionally, but not shown in Figure \ref{fig:JClientConstructor}, some more XMPP features which the library supports, are enabled by the \texttt{init()} function (line 13). This includes, for example, Service Discovery and Roster Management.

\paragraph{}
The function \texttt{connect()}, called in line 5 in Figure \ref{fig:usageExample}, establishes the connection to the XMPP server and creates a new thread that receives incoming TCP/IP packets and hands them to the XML parser of Iksemel.

\paragraph{}
The following interaction with the XMPP server is completely event driven. This is why the example in Figure \ref{fig:usageExample} is not very useful as it is: An application using the Library like this will never now that anything happened to the XMPP connection. The solution is to use the event notification interfaces the Library offers.

\subsection{Event Notification Interfaces}
\label{sec:interfaces}
\paragraph{}
The client library offers a simple yet powerful public interface that enables programs to receive events from an XMPP server with approximately less than 10 additional lines of code. The supported events are:
\begin{itemize}
\item Established and canceled (also lost) connection
\item Incoming messages
\item Incoming Info/Query (IQ) stanzas
\item Incoming presence stanzas
\item Incoming subscription stanzas
\end{itemize}
\paragraph{}
With it, all the possible stanza types that XMPP Core supports are covered. A namespace filter can be utilised for IQ events. In addition to these, the library offers the following interfaces which in turn make heavy use of the previously named ones:
\begin{itemize}
\item Roster Management
\item Service Discovery
\end{itemize}
\paragraph{}
Both of these are built on IQ stanzas and use the namespace filtering mechanism mentioned above.


\paragraph{}
The event notification scheme used by the Client Library follows the so-called Observer Pattern \cite{wikipedia003}. An object interested in XMPP events needs to indicate this to the client library by registering with it as an appropriate event handler. When a stanza arrives from the server, the library calls a special function in every registered object. Figure \ref{fig:observer1} shows an example sequence of registering with the Client Library and being called back by it later on.

\begin{figure}[H]
\begin{center}
\includegraphics{fyp.7}
\end{center}
\caption{Sequence Diagram Illustrating the Observer Pattern}
\label{fig:observer1}
\end{figure}
\paragraph{}
The following interfaces are available for use by an object. Each interface is defined in their own header file for inclusion by interested classes. The virtual function definded therein and listed here have to be reimplemented by the listener objects in order to actually receive the occuring events they intend to listen for.

\paragraph{ConnectionListener}
To be notified about connection-related events, an object must be derived from \texttt{ConnectionListener}. It has to be added to the listener queue by calling
\begin{flushleft}
\texttt{JClient::registerConnectionListener( ConnectionListener* cl );}
\end{flushleft}
After establishing an connection to the XMPP server and successfully authenticating with it, the function
\begin{flushleft}
\texttt{virtual void onConnect() \{\};}
\end{flushleft}
is called in any registered object. Likewise, as soon as a disconnect is detected, either due to a closed stream or a lost TCP/IP connection, 
\begin{flushleft}
\texttt{virtual void onDisconnect() \{\};}
\end{flushleft}
is called.

\paragraph{MessageHandler}
In XMPP, Messages contain the text of a conversation between client users, usually humans. By registering a \texttt{MessageHandler}-derived object with the Library, a program will receive such incoming messages for further processing. Registration happens by calling
\begin{flushleft}
\texttt{void registerMessageHandler( MessageHandler* mh );}
\end{flushleft}
Upon an incoming Message stanza, \texttt{handleMessage()} is called, which has the following signature:
\begin{flushleft}
\texttt{virtual void handleMessage( iksid* from, iksubtype type, const char *msg ) \{\};}
\end{flushleft}
The processing object has immediate knowledge of sender (\texttt{iksid* from}), type (\texttt{iksubtype type}) and content (\texttt{const char *msg}) of the incoming message.

\paragraph{IqHandler}
IQ stanzas usually contain control commands that are evaluated by the client software of an entity and not shown to the user. By registering an \texttt{IqHandler}-derived object with the Library, a program will receive such incoming IQ stanzas for further processing. Registration happens by calling
\begin{flushleft}
\texttt{void registerIqHandler( IqHandler* ih, std::string xmlns );}
\end{flushleft}
Upon an incoming IQ stanza, \texttt{handleIq()} is called, which has the following signature:
\begin{flushleft}
\texttt{virtual void handleIq( const char* xmlns, ikspak* pak ) \{\};}
\end{flushleft}
The processing object has immediate knowledge of the XML namespace (\texttt{const char* xmlns}) and, additionally, access to the complete packet's content by accessing \texttt{ikspak* pak}.

\paragraph{PresenceHandler}
Presence stanzas inform an entity about the current status of a remote entity. Presence stanzas are broadcasted when the status changes. By registering a \texttt{PresenceHandler}-derived object with the Library, a program will receive such incoming presence stanzas for further processing. Registration happens by calling
\begin{flushleft}
\texttt{void registerPresenceHandler( PresenceHandler* ph );}
\end{flushleft}
Upon an incoming Presence stanza, \texttt{handlePresence()} is called, which has the following signature:
\begin{flushleft}
\texttt{virtual void handlePresence( iksid* from, iksubtype type, ikshowtype show, const char* msg ) \{\};}
\end{flushleft}
The processing object has immediate knowledge about sender (\texttt{iksid* from}), type (\texttt{iksubtype type}), display type (\texttt{ikshowtype show}) and content (\texttt{const char *msg}) of the incoming presence stanza.

\paragraph{SubscriptionHandler}
Subscription stanzas are a special case of Presence stanzas. However, their purpose is different from normal Presence, which is why they are handled separately. Subscription stanzas of different types are exchanged when client A wishes to add client B to its roster. By registering a \texttt{SubscriptionHandler}-derived object with the Library, a program will receive such incoming subscription stanzas for further processing. Registration happens by calling
\begin{flushleft}
\texttt{void registerSubscriptionHandler( SubscriptionHandler* mh );}
\end{flushleft}
Upon an incoming Subscription stanza, \texttt{handleSubscription()} is called, which has the following signature:
\begin{flushleft}
\texttt{virtual void handleSubscription( iksid* from, iksubtype type, const char *msg ) \{\};}
\end{flushleft}
The processing object has immediate knowledge about sender (\texttt{iksid* from}), type (\texttt{iksubtype type}) and content (\texttt{const char *msg}) of the incoming message.

\paragraph{Additional XMPP Abtraction Interfaces}
As mentioned above, the Library offers two additional interfaces which enable an application written against the XMPP Client Library to handle protocol extension in a abstracted way. On one hand side, the \texttt{Roster} class handles the contacts which are stored on the server-side contact list (the roster). It offers an extensive interface for roster management. The second interface is an abstraction of the Service Discovery extension. It is implemented in the \texttt{Disco} class.

\paragraph{}
Both of them use the \texttt{IqHandler} interface discussed above. They are written in a modular fashion and can serve as examples for further imlementations of protocol extensions.

\subsection{The Feeder Library}
\label{sec:feederlib}
\paragraph{}
The Feeder Library is separated from the Data Gatherer -- as shown in Figure \ref{fig:feeder} on Page \pageref{fig:feeder} -- to allow for re-use. The functionality the Feeder Library offers is completely independent from the format of the data that is sent over the wire and processed by the Worker. It makes direct use of the XMPP Client Library for the XMPP communication.

\paragraph{}
The purpose of the Feeder Library is, on one hand side, to accept data packets from the Feeder and forward them to a Worker. On the other hand side, it receives result packets from the Workers, evaluates them, stores statistical information about the processed packet in a database and finally passes the result packet to the Feeder.

\paragraph{}
Aditionally, the Feeder Library is responsible for keeping track of available Workers. When a Worker connects to the XMPP server, it announces its availability to all other entities that are listed in its roster -- to be exact, the XMPP server does so. Since the Feeder is on the roster of every Worker (this is a requirement), it gets notified about the newly available Worker. Using Service Discovery, the Feeder Library finds out whether a newly connected entity is capable of processing the data offered by the Data Gatherer, i.e. whether it is really a Worker and compatible. The addresses of all these Workers, together with their current status, are kept in a map. With every status change of a Worker this map is updated. A data packet arriving from the Feeder for distribution is sent to the first Worker available.

\paragraph{}
The inheritance diagram for the main class of the Feeder Library -- \texttt{Feeder} -- is shown in Figure \ref{fig:inhfeeder}.

\begin{figure}[H]
\begin{center}
\includegraphics{fyp.12}
\end{center}
\caption{Inheritance Diagram for Class \texttt{Feeder}}
\label{fig:inhfeeder}
\end{figure}

\texttt{Feeder} generally has to know about connection state changes so it inherits from \texttt{ConnectionListener}. Furthermore, to know about available Workers it implements the \texttt{RosterListener} interface. The Feeder determines via Service Discovery whether an entitiy listed on its roster is capable of processing data, which is why it implements the \texttt{DiscoHandler} interface. Last but not least, since data as well as result packets are wrapped into IQ stanzas, the \texttt{IqHandler} interface is necessary.

\paragraph{}
The communication between Data Gatherer and Feeder Library happens over a special interface following the following algorithm.

\begin{enumerate}
\item The Feeder Library starts by polling for data packets until the Data Gatherer has no more data available.
\item As soon as data is available again, the Data Gatherer starts pushing data packets until all the available Workers are busy processing data.
\item The Feeder Library starts polling -- see step 1.
\end{enumerate}

\paragraph{}
It also would have been possible to use either of the methods exclusively, but for reasons of efficiency in the communication between Data Gatherer and Feeder Library this approach was chosen. The justification is further discussed in Section \ref{sec:designjustification}.

\paragraph{}
The Data Gatherer has to implement the \texttt{PollHandler} interface and register the corresponding object by calling:
\begin{flushleft}
\texttt{void registerPollHandler( PollHandler* ph );}
\end{flushleft}
The actual poll happens by calling the \texttt{poll()} function:
\begin{flushleft}
\texttt{virtual char* poll() \{\};}
\end{flushleft}


\subsection{The Worker Library}
\label{sec:workerlib}
\paragraph{}
Like Data Gatherer and Feeder Library, Algorithm Implementation and Worker Library are separated for easy re-use (Figure \ref{fig:worker} on Page \pageref{fig:worker}). Again, the Worker Library's implementation is independent from type and format of data received from the Feeder and passed to the Algorithm Implementation. Also, the algorithm used to process the data does not influence the Worker Library.

\paragraph{}
The Worker Library offers an interface for the Algorithm Implementation to implement. An object derived from \texttt{DataHandler} has to be registered with the Library by calling:
\begin{flushleft}
\texttt{void registerDataHandler( DataHandler* dh );}
\end{flushleft}
A data packet is passed to the registered object by calling:
\begin{flushleft}
\texttt{virtual void data( const char* data ) \{\};}
\end{flushleft}

\paragraph{}
On startup, the Worker Library establishes a connection to the XMPP server and announces its availability to the Feeder. After that it just waits for a data packet to arrive. When such a packet is received, an acknowledgement is sent to prevent further packets from being sent to this node. The packet is then passed to the Algorithm Implementation for processing using the registered DataHandler. When the Algorithm finishes its processing it calls:
\begin{flushleft}
\texttt{void result( ResultCode code, const char* result );}
\end{flushleft}
\paragraph{}
The Worker Libary constructs a result packet and sends it off to the Feeder.

\paragraph{}
The inheritance diagram for the main class of the Worker Library -- \texttt{Worker} -- is shown in Figure \ref{fig:inhworker}.

\begin{figure}[H]
\begin{center}
\includegraphics{fyp.13}
\end{center}
\caption{Inheritance Diagram for Class \texttt{Worker}}
\label{fig:inhworker}
\end{figure}

\paragraph{}
\texttt{Worker} generally has to know about connection state changes so it inherits from \texttt{ConnectionListener}. Furthermore, to know about the presence of a Feeder and to subscribe to it, the \texttt{RosterListener} interface is implemented. Since data as well as result packets are wrapped into IQ stanzas, the \texttt{IqHandler} interface is necessary.


\subsection{Example Programs}
\label{sec:samples}
\paragraph{}
A set of sample programs has been implemented that demonstrates the usage of the previously discussed libraries. Sample Feeder and Worker together find prime factors of positive integers.

\paragraph{}
The Feeder is simply a counter which is incremented with every \texttt{poll()}.

\paragraph{}
The Worker contains the implementation of the Direct Search Factorization algorithm \cite{mathworld001,wikipedia007}. It tries to find prime factors by simply trying every possible (and reasonable) combination of primes.


\subsection{Statistics Viewer}
\paragraph{}
The Statistics Viewer is a web-based (written in PHP\footnote{PHP has its homepage at href{http://php.net}{http://php.net}}) means to access the statistical data produced by the Feeder in a convenient and user friendly way. In contrast to the previously discussed components, the Statistics Viewer has not been finished.


\subsection{Feeder Front End}
\paragraph{}
The Feeder Front End enables owners or operators of grids to access the Feeder and query it for its run-time parameters. Like the Statistics Viewer, the Feeder Front End has not been finished. It is supposed to use an extension to XMPP called Ad-hoc Commands which is defined in JEP-0050 \cite{jep0050}.

\paragraph{}
JEP-0050: Ad-hoc Commands offers users a way to execute commands on a remote entity. This does not mean the execution of arbitrary programs on the entity's machine, but usage of pre-defined commands within the entity. For example, a service could offer a command that restarts the service or shuts it down.

\paragraph{}
Similar to what the Worker Library does to announce its capability to process data, an entity offering ad-hoc commands and therefore implementing JEP-0050, announces this capability using Service Discovery (as described in Section \ref{sec:disco}).

\section{Results and Discussion}
\label{sec:discussion}

\subsection{Implementation Status}

\begin{table}[H]
\begin{tabularx}{\linewidth}{lc}
\toprule
\textbf{Task} & \textbf{Status} \\
\midrule
\endhead
Database           & finished \\
XMPP Client Lib    & finished \\
Worker             & nearly finished \\
Feeder             & nearly finished \\
Sample Application & started \\
Statistics Viewer  & started \\
Feeder Frontend    & started \\
API Docs           & finished \\
User Manuals       & started \\
\bottomrule
\end{tabularx}
\caption{Implementation Status}
\label{tab:impstatus}
\end{table}


\subsection{Extensibility}
\label{sec:extensibility}

\subsubsection{Load Balancing}
\paragraph{}
As discussed in Section \ref{sec:addressing} and \ref{sec:feederlib}, the Feeder Library stores the full JID for distinction between Workers. A very interesting outcome of this restriction-less design is the possibility to add more than one XMPP server to the cluster and thereby effectively create a load-balanced XMPP server cluster. The limit then becomes the memory of the machine running the Feeder -- it must store the addresses of all the Workers. But even the Feeder is -- sort of -- distributable. It is possible to have more than one Feeder running in an XMPP network.

\subsubsection{Dynamic Cluster}
\paragraph{}
The design of XMPP and the Feeder Libraries allows for addition and removal, respectively, of nodes at run-time. This means that the size of the cluster is fully dynamic. Use cases include:
\begin{itemize}
\item A cluster distributed over the Internet. Similar to Seti@Home, users can connect to the cluster and contribute their machine's processor time at their will.
\item In a university, school or company, machines could be added to the cluster when they are idle. As soon as a machine is needed for a different task, it is temporarily removed from the cluster. This could happen automated and without user intervention if the Worker process ran in the background all the time.
\end{itemize}

\subsubsection{Node Location}
\paragraph{}
The distributability of XMPP servers and nodes as well as the use case from the previous section imply that there is no restriction on the distance between nodes and servers. Of course, greater distances add to the delay of transmissions. But the processing time of data packets is considered high enough to compensate for this.

\subsection{Design Justification}
\paragraph{}
-- for feeder <--> feeder library communication

\subsection{Disadvantages of XMPP}
\paragraph{}
-- overhead of XML


\section{Conclusions and Recommendations for Further Work}
\paragraph{}

\subsection{Conclusions}
\paragraph{}



\subsection{Recommendations for Further Work}
\paragraph{}
Possible further enhancements of the framework include:
\begin{description}
\item[Redundancy] The Feeder Library should send packets to more than one Worker. Incoming results should be compared and in case of differences either another Worker is asked to process the packet again, or if only one out of many Workers replies with an different result, the majority "wins".
\item[Packet Tracking] The Feeder Library should keep track of sent data packets and re-send those which yield no result.
\item[Encryption] If used in an area dealing with sensitive data, encryption should be used to prevent sniffing of data sent over the wire.
\item[CPU monitor] Using a CPU monitor, a Worker could run on an ordinary desktop machine and use the CPU as long as no other tasks are pressing.
\item[Build a Grid] Using this framework, a real grid could be set up that works on a real problem.
\end{description}


\clearpage
\section{Project Planning}

\subsection{Initial Schedule}
\begin{table}[H]
\begin{tabularx}{\linewidth}{llcc}
\toprule
\textbf{} & \textbf{Task} & \textbf{Dependency} & \textbf{Time in Weeks} \\
\midrule
\endhead
A & Requirements                       & --      & 2  \\
B & Design::Worker                     & A       & 6  \\
C & Design::Feeder                     & A       & 6  \\
D & Design::Sample Application         & A       & 2  \\
E & Design::Database                   & A       & 1  \\
F & Design::Statistics Viewer          & A       & 2  \\
G & Design::Feeder Frontend            & A       & 2  \\
H & Implementation::Database           & E       & 1  \\
I & Implementation::Worker             & B       & 7  \\
J & Implementation::Feeder             & C, H    & 7  \\
K & Implementation::Sample Application & D       & 9  \\
L & Implementation::Statistics Viewer  & F, K    & 3  \\
M & Implementation::Feeder Frontend    & G, J    & 3  \\
N & Documentation::API Docs            & I, J    & 5  \\
O & Documentation::User Manuals        & K, L, M & 11 \\
\bottomrule
\end{tabularx}
\caption{Action Plan}
\label{tab:actionplan}
\end{table}

\begin{figure}[H]
\begin{flushleft}
\includegraphics{fyp.1}
\end{flushleft}
\caption{Proposed Schedule $\cdot$ Semester 1}
\label{fig:gantt1}
\end{figure}

\begin{figure}[H]
\begin{flushleft}
\includegraphics{fyp.2}
\end{flushleft}
\caption{Proposed Schedule $\cdot$ Semester 2}
\label{fig:gantt2}
\end{figure}

\begin{table}[H]
\begin{tabularx}{\linewidth}{cX}
\toprule
\textbf{Milestone} & \textbf{Description} \\
\midrule
\endhead
M1 & Interim Report finished by 9th November 2004 \\
M2 & Requirements and Design finished by mid December 2004 \\
M3 & Implementation of core (Worker and Feeder) libraries finished by end February 2005 \\
M4 & Implementation of additional objectives (Feeder Frontend, Statistics Viewer) and documentation finished by mid of March 2005 \\
M5 & Final Report finished by 26th April 2005 \\
\bottomrule
\end{tabularx}
\caption{Milestones}
\label{tab:milestones}
\end{table}


\subsection{Revised Schedule}
\subsection{Revised Schedule}
\paragraph{}
The following is the project schedule as of week 1, semester 2. The XMPP Client Library is a new objective. Due to the evolutionary prototyping development approach taken, design and implementation happen mostly in parallel.

\begin{table}[H]
\begin{tabularx}{\linewidth}{llcc}
\toprule
\textbf{} & \textbf{Task} & \textbf{Dependency} & \textbf{Time in Weeks} \\
\midrule
\endhead
A & Requirements                       & --      & 2  \\
B & Design::XMPP Client Lib            & A       & 11 \\
C & Design::Worker                     & A       & 5  \\
D & Design::Feeder                     & A       & 5  \\
E & Design::Sample Application         & A       & 2  \\
F & Design::Database                   & A       & 1  \\
G & Design::Statistics Viewer          & A       & 1  \\
H & Design::Feeder Front End           & A       & 1  \\
I & Implementation::Database           & F       & 1  \\
J & Implementation::XMPP Client Lib    & B       & 12 \\
K & Implementation::Worker             & C, J    & 3  \\
L & Implementation::Feeder             & D, I, J & 3  \\
M & Implementation::Sample Application & E, K, L & 4  \\
N & Implementation::Statistics Viewer  & G, I    & 1  \\
O & Implementation::Feeder Front End   & H, L    & 1  \\
P & Documentation::API Docs            & J, K, L & 13 \\
Q & Documentation::User Manuals        & M, N, O & 5  \\
\bottomrule
\end{tabularx}
\caption{Revised Action Plan}
\label{tab:rev_actionplan}
\end{table}

\begin{figure}[H]
\begin{flushleft}
\includegraphics{fyp.3}
\end{flushleft}
\caption{Revised Schedule $\cdot$ Semester 1}
\label{fig:rev_gantt1}
\end{figure}

\begin{figure}[H]
\begin{flushleft}
\includegraphics{fyp.4}
\end{flushleft}
\caption{Revised Schedule $\cdot$ Semester 2}
\label{fig:rev_gantt2}
\end{figure}

\begin{table}[H]
\begin{tabularx}{\linewidth}{cX}
\toprule
\textbf{Milestone} & \textbf{Description} \\
\midrule
\endhead
M1 & Interim Report finished by 9th November 2004 \\
M2 & Requirements and overall Design finished by week 6 of semester 2 \\
M3 & Implementation of core objectives (XMPP Client Library, Worker and Feeder Libraries) finished by week 5 of semester 2 \\
M4 & Implementation of additional objectives (Feeder Front End, Statistics Viewer) and documentation finished by week 7 of semester 2 \\
M5 & Final Report finished by 26th April 2005 \\
\bottomrule
\end{tabularx}
\caption{Revised Milestones}
\label{tab:rev_milestones}
\end{table}


\subsection{Final Schedule}
\subsection{Final Schedule}
\paragraph{}
The following is the final schedule as of week 9, semester 2. It shows how time has been spent while working on the project. Reasons for delay and failure to meet some of the objectives have been given in Section \ref{sec:conclusions}.

\begin{table}[H]
\begin{tabularx}{\linewidth}{llcc}
\toprule
\textbf{} & \textbf{Task} & \textbf{Dependency} & \textbf{Time in Weeks} \\
\midrule
\endhead
A & Requirements                       & --      & 2  \\
B & Design::XMPP Client Lib            & A       & 16 \\
C & Design::Worker                     & A       & 7  \\
D & Design::Feeder                     & A       & 7  \\
E & Design::Sample Application         & A       & 2  \\
F & Design::Database                   & A       & 1  \\
G & Design::Statistics Viewer          & A       & 1  \\
H & Design::Feeder Front End           & A       & 1  \\
I & Implementation::Database           & F       & 1  \\
J & Implementation::XMPP Client Lib    & B       & 17 \\
K & Implementation::Worker             & C, J    & 6  \\
L & Implementation::Feeder             & D, I, J & 6  \\
M & Implementation::Sample Application & E, K, L & 2  \\
N & Implementation::Statistics Viewer  & G, I    & 0  \\
O & Implementation::Feeder Front End   & H, L    & 0  \\
P & Documentation::API Docs            & J, K, L & 17 \\
Q & Documentation::User Manuals        & M, N, O & 0  \\
\bottomrule
\end{tabularx}
\caption{Final Action Plan}
\label{tab:fin_actionplan}
\end{table}

\begin{figure}[H]
\begin{flushleft}
\includegraphics{fyp.5}
\end{flushleft}
\caption{Final Schedule $\cdot$ Semester 1}
\label{fig:fin_gantt1}
\end{figure}

\begin{figure}[H]
\begin{flushleft}
\includegraphics{fyp.6}
\end{flushleft}
\caption{Final Schedule $\cdot$ Semester 2}
\label{fig:fin_gantt2}
\end{figure}


\clearpage
\section{Bibliography}
\bibliography{final-report}

\appendix
\appendixpage
\addappheadtotoc

\section{Installation Instructions}
\subsection{}

\section{Selected Source Code}



\end{document}
