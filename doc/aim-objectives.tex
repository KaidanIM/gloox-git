\section{Aims and Objectives}

\subsection{Aims}
\paragraph{}
The main technical aim of this project is to design and implement a prob\-lem-in\-de\-pen\-dent software framework that helps setting up a grid, supporting an arbitrary number of machines. Computationally expensive tasks can then be distributed to the nodes of this grid. Communication of these nodes with the coordinating master happens by means of the Extensible Message and Presence Protocol (XMPP).
\paragraph{}
Additionally, this project has educational aims. Its purpose is to help the author with getting more comfortable in dealing with larger projects. It serves as an example of work which has to be carried out with only little assistance.

\subsection{Objectives}
\label{sec:objectives}
\paragraph{}
The project is split into several sub-tasks which mostly depend on one another. The particular dependencies, as fixed for the final schedule, are listed in Figure \ref{tab:fin_actionplan} on page \pageref{tab:fin_actionplan}.

\paragraph{}
The following subsidiary objectives are considered part of the project aim:
\begin{enumerate}
\item \label{item:req} Requirements Specification for the project as a whole
\item \label{item:design} Design of the individual parts of the project (Worker, Feeder, sample implementations, Statistics Viewer, Feeder Front End)
\item \label{item:db} Defining the database structure
\item \label{item:client} Writing the XMPP Client Library
\item \label{item:worker} Writing the Worker Library
\item \label{item:feeder} Writing the Feeder Library
\item \label{item:samplF} Writing a Sample Worker implementation
\item \label{item:samplW} Writing a Sample Feeder implementation
\item \label{item:testing} Testing
\item \label{item:frontend} Writing the Feeder Front End
\item \label{item:viewer} Writing the Statistics Viewer
\item \label{item:manual} Writing User Manuals and generating API documentation
\end{enumerate}
Items \ref{item:req} through \ref{item:testing} are considered core parts of the project. Therefore, most attention will be paid to these tasks. As explained in Section \ref{sec:testing}, testing happens in parallel to the programming. There is no extra testing time planned.
