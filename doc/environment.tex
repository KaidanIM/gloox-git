\section{Development Environment}
\paragraph{}

\subsection{Operating System}
\paragraph{}
The Operating System the development of this framework takes place on is a home-grown Linux system. During the development cycle it was made sure that the software compiles on other Linux boxes as well.

\subsection{Dependencies}
\label{sec:depend}

\subsubsection{Compile-Time Dependencies}
\paragraph{}
The most important dependency is the Iksemel library, written by Gurer Ozen. Iksemel contains XML parsers (DOM as well as SAX) and offers many XMPP related functions. It was chosen as parsing library for this project due to its simplicity and efficiency of usage, the small size and its self-containment.

\paragraph{}
Early versions of the XMPP Client Library developed in this project depended on the Iksemelmm library which is a C++ wrapper around Iksemel. To reduce the number of external dependencies, the most important parts (namely the class Tree, Parser and Stream) of it where added to the XMPP Client Library.


\subsubsection{Run-Time Dependencies}
\paragraph{XMPP Server}
The central workhorse, as far as the communication between Worker and Feeder is concerned, is the XMPP server. For this project the jabberd2\footnote{Jabberd2 has its home at http://jabberd.jabberstudio.org} server was chosen. Jabberd2 is an open source XMPP compliant server implementation.
\paragraph{}
The XMPP server is responsible for authenticating users and routing stanzas between clients in an XMPP/Jabber network. As mentioned in Section \ref{sec:xmpp}, it also distributes presence information to authorised contacts.

\paragraph{Database}
As a database backend, storing statistical information about the taskshanded out to the Workers, MySQL has been chosen because it is already in use on the machine acting as master in the development infrastructure. Since the storage mechanism is implemented in a modular way, it would be easy to add a PostgreSQL or file-based storage backend.

\paragraph{Webserver}
The webserver running the Statistics Viewer is, in the development infrastructure, Apache httpd. Again, it is used because it already runs on the master machine. In a later production environment, Apache httpd is not necessarily required since PHP, which is used for the implementation of the Statistics Viewer, runs on a large range of webservers.


\subsection{Source Code Versioning}
\paragraph{}
For software developers it is very important to have full control over source code they have written. This becomes more important with a higher number of developers involved in a single project. In this case usually a Revision Control System (RCS) is used to keep track of code changes. But even for a one-person project a RCS has several advantages.

\begin{itemize}
\item backup, if revision control system is kept on different machine
\item history of modifications including log messages
\item recovery of older versions
\item branching and merging of changes
\end{itemize}

\paragraph{}
From the very first this project used a RCS called Subversion\footnote{Subversion lives at http://subversion.tigris.org}. It has been choosen due to positive experience in earlier and current projects.

\paragraph{}
The Subversion repository for this project can be found at \\ \texttt{http://svn.camaya.net/uni/fyp-lsbu/}. The version accessible via HTTP is the current one. Older revisions can be accessed with an appropriate Subversion client. The repository hosts the complete source code and all the documentation written during this project's life-span.


\subsection{Build System}
\paragraph{}
The Build System helps users and developers building binary executables as well as documentation from source code. The responsibilities of the Build System include:

\begin{itemize}
\item Checking for availability of dependecies
\item Compiling source files using an appropriate compiler
\item Installing executables and data files to their destination directories
\end{itemize}

\subsubsection{Source Code Compilation}
\paragraph{}
For this project a combination of the tools autoconf\footnote{Autoconf's homepage is at http://www.gnu.org/software/autoconf/} and automake\footnote{Automake's homepage is at http://www.gnu.org/software/automake/} is used. Of course, The GNU Compiler Collection  takes care of the final compilation.

\paragraph{autoconf}
Autoconf is used to create a \texttt{configure} script. This contains various checks for more or less obvious dependencies, including those listed in Section \ref{sec:depend}.

\paragraph{automake}
Automake is used to create target-platform specific \texttt{Makefile}s from their corresponding templates (called \texttt{Makefile.am}). A \texttt{Makefile} the includes a set of rules to build the source code, i.e. the actual commands and the order they are executed in.

\subsubsection{API Documentation}
\paragraph{}
The API documentation is created by Doxygen\footnote{Doxygen's homepage is at http://www.doxygen.org}. It extracts comments from the source files and builds a nice documentation in HTML format from it.


\subsubsection{Project Documentation}
\paragraph{}
The project documentation is written in \LaTeX.
