\section{Introduction}
\paragraph{}
In almost every branch of science data is either collected or created that has to be processed in some way. Sometimes a single machine is unable to keep up with the sheer amount of data and/or the demand for computing power to handle this data in a timely manner. A possible solution to this problem is, of course, to invest in faster hardware. This is a somewhat limited approach. At some point it is simply no longer possible to buy a faster (single-CPU) computer or, more likely for the average company, the costs for acquirement, installation and maintenance are much too high and outweigh the benefits. A cheaper solution would be to utilise a cluster or a grid.
\paragraph{}
A cluster, as \citeasnoun*{wikipedia001} explains, is an arbitrary number of computers, connected over some kind of network and offering some sort of parallelisation. To the user a cluster usually looks like a single computer. The software takes care of distributing tasks, data or algorithms. Actually, according to \citeasnoun*{wikipedia002} and \citeasnoun*{top500}, even today's supercomputers are clusters. Allowedly with litteraly thousands of CPUs and specially crafted interconnects to make fast communtication between CPUs and memory possible.
\paragraph{}
Grids differ from clusters in that they are a heterogenous network of computers, while clusters are homogenous. Also, grids are more loosely connected than clusters and can utilise a user's desktop system or office machines \cite{wikipedia005}. To form of a grid, particular software is needed that keeps track of nodes, distributes data and manages the results returned.
\paragraph{}
The Extensible Messaging and Presence Protocol (XMPP) defines a means for interprocess communication based on the Extensible Markup Language (XML). It evolved from the Jabber protocol developed in an Internet community since 1999. In May 2004 Internet Engineering Steering Group (IESG) approved the core parts of XMPP \cite{xmpp-core,xmpp-im} as Proposed Standards and in October 2004 the protocols were published as Request for Comments (RFC).
\paragraph{}
The intention of this project is to utilise XMPP to combine the computing power of a number of networked machines to solve a specific problem. XMPP will be used to control data flow from and to the nodes.
