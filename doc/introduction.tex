\section{Introduction}
\paragraph{}
Almost every branch of science either collects or creates data that has to be processed in some way. Sometimes a single machine is unable to keep up with the sheer amount of data and/or the demand for computing power to handle this data in a timely manner. A possible solution to this problem is, of course, to invest in faster hardware. This is a somewhat limited approach. At some point it is simply no longer possible to buy a faster (single-CPU) computer or, more likely for the average company, the costs for acquirement, installation and maintenance are much too high and outweigh the benefits. A cheaper solution would be to set up a cluster.
\paragraph{}
A cluster, as \citeasnoun*{wikipedia001} explains, is an arbitrary number of conventional computers, connected over a network and possibly running special software. To the user a cluster can either look like a single computer by simulating a multi-processor machine or offer parallelisation for special demands such as compiling software or hosting network services. Actually, according to \citeasnoun*{wikipedia002} and \citeasnoun*{top500}, even today's supercomputers are clusters. Allowedly with litteraly thousands of CPUs and specially crafted interconnects to make fast communtication between CPUs and memory possible.
%\paragraph{}
%Problems that arise when running a cluster include the power consumption (and therefore heat production) and the space needed to host it. The solutions to both problems are a bit contradictive since minimising required space means more powerful cooling is necessary. Additionaly, to gain maximum performance, signal paths must be kept as short as possible to keep latency down. To circumvent the former two probblems, by increasing the latter, though, a decentralized structure must be used. Several possibilities exist, one being to find volunteers who spend some of the processing power of their home machines to help solve, for example, a mathematical problem. Also, this seems like the cheapest way to operate a cluster. A popular example for this approach is \citename*{seti}.
\paragraph{}
To make use of a cluster, a particular software is needed to distribute fractions of data to the nodes and to manage the results returned. Such software is responsible for supplying clients with new data to process once a result is delivered back to the originating machine.
\paragraph{}
The Extensible Messaging and Presence Protocol (XMPP) defines a means for interprocess communication based on the Extensible Markup Language (XML). It evolved from the Jabber protocol developed in an internet community since 1999. In May 2004 Internet Engineering Steering Group (IESG) approved the core parts of XMPP \cite{xmpp-core,xmpp-im} as Proposed Standards and in October 2004 the protocols were published as Request for Comments (RFC)\footnote{RFC 3920 and RFC 3921}.
\paragraph{}
The intention of this project is to utilise XMPP to combine the computing power of a number of networked machines to solve a specific problem. XMPP will be used to control data flow from and to the nodes.
