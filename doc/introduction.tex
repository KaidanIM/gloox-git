\section{Introduction}
\paragraph{}
In almost every branch of science data is either collected or created that has to be processed in some way. Sometimes a single machine is unable to keep up with the sheer amount of data and/or the demand for computing power to handle this data in a timely manner. A possible solution to this problem is, of course, to invest in faster hardware. This is a somewhat limited approach. At some point it is simply no longer possible to buy a faster (single-CPU) computer or, more likely for the average company, the costs for acquirement, installation and maintenance are much too high and outweigh the benefits. A cheaper solution would be to utilise a cluster or a grid.
\paragraph{}
A cluster, as \citeasnoun*{wikipedia001} explains, is an arbitrary number of computers, connected over some kind of network and offering some sort of parallelisation. To the user a cluster usually looks like a single computer. The software running on these machines takes care of distributing tasks, data or algorithms. Actually, according to \citeasnoun*{wikipedia002} and \citeasnoun*{top500}, even today's supercomputers are clusters, with literally thousands of CPUs and specially crafted interconnects to make fast communtication between CPUs and memory possible. Clusters are homogenous systems, which means they are made up of identical machines. They are usually located in some sort of data center, consist of a fixed number of machines and are owned by a single organisation. Clusters are mostly used for commercial or academic applications.
\paragraph{}
Grids differ from clusters in that they can consist of machines of different platforms -- they build a heterogenous network of computers. Also, grids are more loosely connected than clusters. For example, in a company's office they can utilise the user's desktop systems \cite{wikipedia005}. A machine can become a member of a grid at any time and can also leave the grid at any time. To form a grid, particular software is needed that keeps track of connectiong and disconnecting nodes, distributes data and manages the results returned. A grid is not fixed to a particular location or owner. Some grids set up today span thousands of machines communicating via the Internet.
\paragraph{}
The Extensible Messaging and Presence Protocol (XMPP) defines a means for interprocess communication based on the Extensible Markup Language (XML). It evolved from the Jabber protocol developed in an Internet community since 1999. In May 2004 Internet Engineering Steering Group (IESG) approved the core parts of XMPP \cite{xmpp-core,xmpp-im} as Proposed Standards and in October 2004 the protocols were published as Request for Comments (RFC).

\begin{figure}[H]
\begin{center}
\includegraphics{fyp.0}
\end{center}
\caption{Project Architecture}
\label{fig:architectureIntro}
\end{figure}

\paragraph{}
The intention of this project is to utilise XMPP to combine the computing power of a number of networked machines to solve a specific problem. Figure \ref{fig:architectureIntro} shows the components of this project. A so-called Feeder is responsible for gathering or creating data that is processed by a number of Workers. XMPP is used as the underlying communication protocol between Feeder and Workers. An XMPP server, which is a $3^{rd}$-party product, routes the XMPP traffic between nodes in the grid. A database is used to store statistical information about results which can be viewed using the Statistics Viewer.
